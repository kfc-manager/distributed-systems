\documentclass{article}
\usepackage{graphicx}
\usepackage{natbib}
\usepackage{listings}
\usepackage{amsmath}
\usepackage{amssymb}
\usepackage{mathtools}
\usepackage{subfiles}
\usepackage{geometry}
 \geometry{
 a4paper,
 total={170mm,257mm},
 left=20mm,
 top=20mm,
 }

\lstdefinestyle{CStyle}{
  basicstyle=\ttfamily,
  breakatwhitespace=false,
  breaklines=true,
  captionpos=b,
  keepspaces=true,
  numbers=left,
  numbersep=5pt,
  showspaces=false,
  showstringspaces=false,
  showtabs=false,
  tabsize=2,
}

\lstset{style=CStyle}

\title{Distributes Systems}
\author{Kilian Calefice (796461)}
\date{20.04.2024}

\begin{document}

\maketitle

\section*{1. Parameter Passing}

a) In the lecture, we discussed the parameter passing approaches Call-By-Value and Call-By-Reference.
Here, we also explore Call-By-Copy/Restore. With Call-By-Value, a copy of the provided parameter value is
given to the callee to operate on. In contrast, with Call-By-Reference, a reference to the variable 
provided as a parameter is given to the callee to operate on. Again, in Call-By-Copy/Restore, parameter
passing is handled by sending a copy of the referenced variable to the callee, and on return replacing
the caller's copy with that modified by the callee.\\
With RPC, all three parameter passing approaches may be used. What will the output of the procedure 
\texttt{MAIN} be (i. e., the value of the array $a$ on line 11) if \texttt{ROTATE} is called with an RPC
protocol using...\\
\\
\begin{itemize}
  \item ...Call-By-Value\\
  \item ...Call-By-Reference\\
  \item ...Call-By-Copy/Restore
\end{itemize}
\begin{lstlisting}[style=CStyle]
procedure ROTATE(var x,y: array[6], var n: integer, var i:integer)
  y[i] <- x[(i + n) mod 6];
  if i < 5 then
    ROTATE(x, y, n, i+1)
  end if
end procedure

procedure MAIN
  var a: array[6];
  a <- [A, B, C, D, E, F];
  ROTATE(a, a, 3, 0);
  print a;
end procedure
\end{lstlisting}
\begin{itemize}
  \item Call-By-Value: [A, B, C, D, E, F]\\
  \item Call-By-Reference: [D, E, F, D, E, F]\\
  \item Call-By-Copy/Restore: [D, E, F, A, B, C]
\end{itemize}

\section*{2. Socket-Programming, Chat - Client-Server}

Implement a chat system, which uses UDP (User Datagram Protocol) in C:\\
\\
a) First the server is communicating with only the client.
\begin{itemize}
  \item The server (\texttt{chat_server.c}) should wait on a port given by the parameter for requests.\\
  \item The client (\texttt{chat_client.c}) sends a message to the server. The server's hostname, 
    portnumber and chatname will be given a parameters.\\
  \item The chat server prints the message, the origin, and the timestamp on \texttt{stdout}.\\
  \item The chat server sends an acknowledgment to the chat client.
\end{itemize}
\textbf{Example:}
\begin{itemize}
  \item The server is called with: \texttt{chat_server portnumber}\\
  \item The client call looks like: \texttt{chat_client host portnumber chatname}\\
  \item Your implementation should first be tested on your \texttt{localhost}. The solution for this
    assignment is working between \texttt{tiree} and \texttt{coll}. Please use the portnumbers between
    20001 and 20020.
\end{itemize}

\section*{3. Chord System}

a) Node 30 needs to resolve the key $k = 16$. Give the sequence of nodes that are visited by this request,
until the node storing $k$ is reached. Also, give the complete finger table of each node storing $k$ is
reached. Also, give the complete finger table of each node visited by this request.\\
\\
b) Now, assume that node 19 has just joined the network. What is the finger table of node 19 after the
join procedure has completed? What other finger tables in the network will actually change (you do not
have to compute the changed finger tables)?\\
\\
c) Assume that two nodes join the network at exactly the same time. In which case can this lead to
inconsistencies in the Chord ring? Name these inconsistencies and briefly illustrate a small example of
the problem.\\
\\
d) In the lecture, part of the an


\end{document}
